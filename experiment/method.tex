\section{実験方法}
\subsection{実験環境}
実験環境として~~~~に示す千葉工業大学2号館3階を用いる.

図を追加する

また,経路追従モジュールと通路分類モジュールの学習データを収集するために,~~~に示すルートを走行する.

\subsection{シナリオの選定}
実験では島田ら用いた50例の中から,22例を選定した.
選定するにあたって,以下の条件を設定した.

1)ロボットが移動困難な~~~に示すルートが含まれないこと.

2)経路追従モジュールができない,その場での旋回が含まれていないこと.

3)通路の分類が困難な~~~に示すルートが含まれないこと.

\subsection{経路追従モジュールの訓練}
実験環境で明示したルートをオンライン学習させながら1週走行する.
走行後はepoch数20でオフライン学習を行う.

\subsection{通路分類モジュールの訓練}
実験環境で明示したルートをROS の navigation パッケージを使用して,経路を1周する.
その際,3つのカメラからそれぞれ画像データを収集しながら走行する.
学習時のパラメータとして,バッチサイズ32,epoch数30とし,コストアプローチに用いた重みは~~~に示す.