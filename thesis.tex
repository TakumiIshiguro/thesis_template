\documentclass[uplatex, a4paper, 12pt, openany, oneside]{jsbook}

\usepackage[dvipdfmx]{graphicx}
\usepackage[dvipdfmx]{color}
\usepackage[dvipdfmx, bookmarks=true, setpagesize=false, hidelinks]{hyperref}
\usepackage{subcaption} 
\usepackage{pxjahyper}
\usepackage{thesis}
\usepackage{here}
\usepackage{url}


\thesis{卒 業 論 文}
\title{
  \centering
    \scalebox{0.8}{視覚と行動のend-to-end学習により}\\
    \vspace{-0.3zh}
    \scalebox{0.8}{経路追従行動を模倣する手法の提案}\\ 
    \vspace{-0.3zh}
    \scalebox{0.7}{-経路選択の成功率向上を意図したネットワークの変更と実験的評価-}\\ 
    % \vspace{-1.0zh}
    \scalebox{0.6}{A proposal for an imitation method of path-tracking behavior}\\ 
    \vspace{-0.6zh}
    \scalebox{0.6}{by end-to-end learning of vision and action}\\ 
    \vspace{-0.6zh}
    \scalebox{0.5}{-Modification of network and experimental evaluation aimed at improving}\\
    \vspace{-0.6zh}
    \scalebox{0.5}{route selection success rate-}\\ 
    \vspace{-10zh}
}
\setlength{\textwidth}{\fullwidth}
\setlength{\evensidemargin}{\oddsidemargin}

\date{\today}
\vspace{-15.0zh}
\teacher{林原 靖男 教授}
\vspace{-15.0zh}
\organization{千葉工業大学 先進工学部 未来ロボティクス学科}
\author{21C1011 石黒 巧}
\vspace{-15zh}

\renewcommand{\baselinestretch}{1.2}
\begin{document}

%% Front Matter
\frontmatter{}
%
\chapter{機能の改善}
\label{chap:method}
%
%\input{method/preface}
%
\section{経路追従モジュール}
%
\section{通路分類モジュール}


%
%% Main Matter
\mainmatter{}
%
\chapter{機能の改善}
\label{chap:method}
%
%\input{method/preface}
%
\section{経路追従モジュール}
%
\section{通路分類モジュール}


%ここにディレクトリのパスを追加していく
\chapter{要素技術}
\label{chap:elemental}
本章では,本研究に関連する要素技術を述べる
\section{メトリックマップに基づくナビゲーション}
メトリックマップに基づくナビゲーションについて述べる.
ナビゲーションを行うためには,LiDARやオドメトリなどのセンサとメトリックマップを用いて,自己位置推定,経路計画を行い,目的地まで自律移動する.
自己位置推定は,ロボットが地図上でどこにいるかを特定するために行い,LiDARやオドメトリデータを用いてAMCL(Adaptive Monte Carlo Localization)などのアルゴリズムを活用する.
次に,グローバルプランニングで現在位置から目的地までの最適経路を計画する.
計画された経路に基づき,ローカルプランニングがロボットの動作をリアルタイムで制御し,動的な障害物や環境の変化に対応する.
センサデータは,動的なコストマップを更新し,環境の変化を即座に反映する.この仕組みにより,静的なマップに基づいたプランニングと動的な環境対応を統合し,安全かつ効率的な自律移動を実現する.
メトリックマップに基づくナビゲーションは,精密な環境情報を活用できる点が利点である.
しかし,高精度な地図やセンサ,計算資源を必要とするため,これらの要件に応じたシステム設計が求められる.
\chapter{先行研究}
\label{chap:prior}
はじめに,視覚からend-to-end学習によってナビゲーションを獲得した,岡田らの手法とFelipeらの手法を述べる.
次に,島田らが提案したトポロジカルマップの形式,単語の組み合わせによる経路の表現であるシナリオについて述べたのち,春山らの視覚に基づいて目的地まで自律移動するシステムについて述べる.
\section{視覚と行動の end-to-end 学習により経路追従行動をオンラ
インで模倣する手法}
岡田らの手法では,メトリックマップに基づく経路追従行動を視覚を入力とした行動へ模倣するために,end-to-end学習を用いた手法を提案している.
ロボットは経路を自律移動するのと同時に学習を行う.

訓練時には,ROS の navigation パッケージを使用して,設定した経路を追従する.
その際,ロボットに取り付けたカメラから取得したRGB画像とルールベース制御器が出力するヨー方向の角速度をペアにして,0.2秒の周期でデータセットに追加する.
次に,このデータセットからバッチサイズ8で教師データを抽出し,end-to-end学習を行う.
このデータ収集から学習までの一連の流れを1ステップと定義している.
収集には3台のカメラを使用することで,データの多様性を高めるとともに過学習を防ぐ効果を狙っている.
左右のカメラ画像に対するヨー方向の角速度には経路復帰を補助するためのオフセット(±0.2rad/s)を加える.

学習器の訓練後は,中央のカメラから得たRGB画像を入力とし,出力されるヨー方向の角速度を用いて経路を追従する.
学習時,学習後ともに並進速度は0.2m/sに固定し,カメラ画像は64×48にリサイズする.
この手法により,学習した経路を,画像のみを入力とした学習器の出力で自律移動できることが確認されている.
\clearpage
\section{条件付き模倣学習}
1 章でも述べたが,Felipe ら\cite{codevilla2018endtoend}は視覚を入力とした end-to-end 学習により自動運転を行う手法において,右折や左折といった行動をネットワークの入力に加えることで,性能が向上することを報告している.
Felipe らは 2 種類のネットワークを提案している.
\figref{fig:felipe_branched}に示す (a) のネットワークは画像を処理する CNN ,そして CNN の出力と観測,目標方向などのコマンドを入力とする全結合層で構成されている.
\figref{fig:felipe_branched}に示す (b) のネットワークは画像を処理する CNN ,そして CNN の出力と観測を目標方向などのコマンドによって全結合層を切り替える構造となっている.
シミュレータと実環境の両方で実験が行われており,どちらも (b) のネットワークがより優れた結果となった.

\begin{figure}[htbp]
  \centering
  \includegraphics[width=140mm]{images/pdf/other/branched.pdf}
  \caption[Two network architectures for command-conditional imitation learning]{Two network architectures for command-conditional imitation learning 
  \protect\linebreak (Quoted from\cite{codevilla2018endtoend})}
  \label{fig:felipe_branched}
\end{figure}

\clearpage

\section{トポロジカルマップとシナリオ}
\subsubsection{トポロジカルマップ}
トポロジカルマップとは,環境をランドマークや特徴的な箇所をノードとし,その繋がりをエッジで表現した地図である.
島田らが提案したトポロジカルマップでは,ノードは通路の特徴的な箇所に配置され,エッジはノード間を接続する.
ノードにはID,通路の特徴(Type),エッジIDと相対角度(Edge)のデータが含まれ,エッジにはIDのみが含まれる.
この形式は道案内に関するアンケート結果に基づいており,人が道案内で「通路の特徴」や「向いている方向」を重視することが明らかになったことから設計された.
\subsubsection{シナリオ}
シナリオは,トポロジカルマップ上の目的地までの経路を「条件」と「行動」の組み合わせで表現する手法である.
「条件」は「次の角」や「突き当たりまで」などを指し,「行動」は「直進」や「右折」などを指す.
この形式はトポロジカルマップ同様に道案内に関するアンケート結果に基づいており,人が「条件」と「行動」を組み合わせて道案内をすることが明らかになったことから設計された.
例として,特定の経路はと表現される.

\clearpage
\section{視覚に基づいて目的地まで自律移動するシステム}
春山らは,カメラ画像とトポロジカルマップから作成されるシナリオに基づいて,目的地まで自律移動するシステムを構築している.
提案されたシステムの概要図を\figref{fig:sys}に示す.

\begin{figure}[htbp]
  \centering
   \includegraphics[width=130mm]{images/pdf/haruyama/system.pdf}
   \caption{Overview of proposed system (Quoted from \cite{haruyama2023})}
   \label{fig:sys}
\end{figure}

1)シナリオを分解し,「条件」と「行動」を抽出するモジュール(以後,シナリオモジュールと呼ぶ)

2)カメラ画像と目標方向を与えることで,経路を追従するモジュール(以後,経路追従モジュールと呼ぶ)

3)カメラ画像から通路の特徴を分類するモジュール(以後,通路分類モジュールと呼ぶ)

の3つのモジュールで構成される.ロボットは下記の a から d の一連の流れにより,指示された経路に沿って目的地まで自律移動する.
\begin{enumerate}
  \item [(a)] 
  トポロジカルマップ上の目的地に応じて,人間が「条件」と「行動」で構成されるシナリオを作成する.
  例えば,図のトポロジカルマップ上でAを目的地とするシナリオは「次の角まで直進.左折.」となる.
  \item [(b)] 
  作成したシナリオをシナリオモジュールへ入力する.
  シナリオモジュールは入力されたシナリオを分解し,「条件」と「行動」を抽出する.
  1つ目の条件と行動のセットは「次の角まで」と「直進」となる.
  この「直進」を目標方向として経路追従モジュールへ与える.
  経路追従モジュールは,カメラ画像と与えられた目標方向に基づいて,経路に沿って直進する.
  \item [(c)] 
  ロボットが角に近づくと,通路分類モジュールがカメラ画像に基づいて通路を「角」と分類し,それをシナリオモジュールに与える.
  シナリオモジュールはそれを基に「次の角まで」という条件を満たしたかを判定する.
  この場合は条件を満たしているため,2つ目の行動である「左折」へ遷移する.
  \item [(d)]「左折」に基づいて,経路追従モジュールは経路に沿って角を左折する.
\end{enumerate}

\subsubsection{シナリオモジュール}
シナリオモジュールは,トポロジカルマップを基に作成されたシナリオから「条件」や「行動」を解釈し,それを分岐路での目標方向に変換して出力する機能を持つ.
トポロジカルマップは,特徴的な通路のノード(青)とそれを繋ぐエッジ(緑)で構成され,ノードにはIDや通路の特徴,接続エッジと方向のデータが含まれている.
シナリオは目的地までの経路を「条件」と「行動」で表現し,例として「三叉路まで直進.右折.突き当たりまで直進.停止」となる.

シナリオの目標方向への変換では,句点ごとに分解し,「条件」と「行動」を抽出して以下の項目に分類する:

通路の特徴(例:「三叉路」「角」)

順番(例:「3つ目の」「2番目の」)

方向(例:「左手に」「右手に」)

行動(例:「右折」「停止」)

例では,「三叉路まで直進」は通路の特徴「三叉路」と行動「直進」に分解される.
この処理を経路全体に対して行い,得られた「行動」を分岐路での目標方向として変換し,経路追従モジュールに渡す.
また,条件の判定には通路分類モジュールを使用する.

\begin{figure}[htbp]
  \centering
   \includegraphics[width=130mm]{images/pdf/haruyama/scenario.pdf}
   \caption{ Example of topological map and created scenario Quoted from \cite{haruyama2023}}
   \label{fig:scenario}
\end{figure}

\clearpage
\subsubsection{経路追従モジュール}
このモジュールは,岡田らの手法から目標方向のデータを加えることで,分岐路で経路を選択し,移動する機能を追加したものである.
ここで目標方向とは,目標とする進行方向(「直進」や「右折」)を表す.
学習時は,カメラ画像とルールベース制御器が出力するヨー方向の角速度,目標方向を 0.2 秒周期でデータセットに加える.
データセットから抽出するバッチサイズや,カメラ画像の解像度は岡田ら手法と同様である.
データセットの収集には藤原らが提案した,データセットに加えるデータの不均衡を改善する手法,学習時に積極的な蛇行する手法を採用する.

\begin{figure}[htbp]
  \centering
   \includegraphics[width=110mm]{images/pdf/haruyama/pathfollow_sys.pdf}
   \caption{Path-following module system Quoted from \cite{haruyama2023}}
   \label{fig:pathfollow}
\end{figure}



\subsubsection{通路分類モジュール}
このモジュールでは,ニューラルネットワークを用いることで,カメラ画像を入力として,通路の特徴を分類する.
データセットの収集をするために,ロボットをルールベース制御器に基づいて走行させる.
その際に,フレーム数 16 ,画像サイズ 64 × 48の連続したカメラ画像と通路の分類ラベルを1組として,0.125 秒周期でデータセットに加える.
通路の分類ラベルのアノテーションはルールベース制御器から出力されるラベルによって自動的に行う.
データセット内の不均衡を改善するために,クラス間のデータ数によって重み付けを行うコストアプローチを導入している.

\begin{figure}[htbp]
  \centering
   \includegraphics[width=130mm]{images/pdf/haruyama/intersection_sys.pdf}
   \caption{Path-following module system Quoted from \cite{haruyama2023}}
   \label{fig:intersection}
\end{figure}

\subsubsection{実ロボットを用いた実験}
実ロボットを用いた実験により,ロボットを目的地まで到達可能か検証されている.
実験では島田ら用いた 50 例のシナリオの中から,7例が用いられており,そのすべてでロボットが目的地へ到達できることが確認されている.


\chapter{機能の改善}
\label{chap:method}
%
%\input{method/preface}
%
\section{経路追従モジュール}
%
\section{通路分類モジュール}


\chapter{新たなシナリオが走行できるか検証}
\label{chap:experiment}
\section{実験装置}
実験には \figref{fig:gamma} に示す icart-mini\cite{icart} をベースに開発したロボットを用いる.
センサとして,単眼のウェブカメラ (サンワサプライ株式会社 CMS-V43BK) を 3 つ,2D-LiDAR(北陽電機 UTM-30LX) を 1 つ,左右のモータにそれぞれパルス付きエンコーダを搭載している.
制御,学習用の PC には GALLERIA GCR2070RGF-QC-G を使用している.
メトリックマップに基づくルールベース制御器には,本学で ROS Navigation stack をもとに開発したorne navigation\cite{orne_nav}を使用する.

\begin{figure}[htbp]
    \centering
     \includegraphics[width=60mm]{images/pdf/haruyama/gamma.pdf}
     \caption{Experimental setup}
     \label{fig:gamma}
\end{figure}
\newpage
\section{実験方法}
\subsection{実験環境}
実験環境として\figref{fig:topo}に示す千葉工業大学2号館3階を用いる.
環境中には,三叉路が6つ,角が3つ,突き当りが4つ含まれている.
また,経路追従モジュールと通路分類モジュールの学習データを収集するために,\figref{}に示すルートを走行する.

\begin{figure}[htbp]
  \centering
  \includegraphics[width=130mm]{images/pdf/ishiguro/topo.pdf}
  \caption{Experimental environment}
  \label{fig:topo}
\end{figure}

\subsection{シナリオの選定}
実験では島田ら用いた50例の中から,28例を選定した.
選定するにあたって,以下の条件を設定した.

\begin{enumerate}
  \item [1)] ロボットが移動困難な\figref{fig:cit3f}に示す部分を走行ルートに含まれないこと.
  \item [2)] 経路追従モジュールができない,「後ろを向く」などのその場での旋回が含まれていないこと.
  \item [3)] 通路の分類が困難な\figref{fig:cit3f}に示す部分が走行ルートが含まれないこと.
\end{enumerate}

\begin{figure}[htbp]
  \centering
  \includegraphics[width=130mm]{images/pdf/ishiguro/cit3f.pdf}
  \caption{Experimenta}
  % \label{fig:topo}
\end{figure}

\subsection{経路追従モジュールの訓練}
\figref{}に示すルートをオンライン学習させながら1週走行する.
オンライン学習で作成したモデルに追加でオフライン学習を行う.
オフライン学習時のデータセットはオンライン学習の際に作成した1週分のデータを用いる.
データセットからはオンライン学習と同様のバッチサイズ8で画像をランダムに取得し,epoch数は20とした.

\subsection{通路分類モジュールの訓練}
\figref{}に示すルートをROS の navigation パッケージを使用して,経路を1周する.
その際,3つのカメラからそれぞれ画像データを収集しながら走行する.
学習時のパラメータとして,バッチサイズ32,epoch数30とし,コストアプローチに用いた重みは~~~に示す.

\subsection{シナリオに基づくナビゲーション}
2 つのモジュールを訓練後,ロボットが目的地まで到達できるか確認する.
実験では,ロボットをシナリオのスタート地点,向きに配置し,シナリオを1例ずつ投入する.
途中で壁に衝突や,経路の選択を誤ることなく自律移動し,目的地で停止した際に成功とする.
\chapter{おわりに}
\label{chap:end}
\section{結論}

%
%% Back Matter
\backmatter{}
%
\chapter{機能の改善}
\label{chap:method}
%
%\input{method/preface}
%
\section{経路追従モジュール}
%
\section{通路分類モジュール}


%

\end{document}
