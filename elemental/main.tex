\chapter{要素技術}
\label{chap:elemental}
本章では,本研究に関連する要素技術を述べる
\section{メトリックマップに基づくナビゲーション}
メトリックマップに基づくナビゲーションについて述べる.
ナビゲーションを行うためには,LiDARやオドメトリなどのセンサとメトリックマップを用いて,自己位置推定,経路計画を行い,目的地まで自律移動する.
自己位置推定は,ロボットが地図上でどこにいるかを特定するために行い,LiDARやオドメトリデータを用いてAMCL(Adaptive Monte Carlo Localization)などのアルゴリズムを活用する.
次に,グローバルプランニングで現在位置から目的地までの最適経路を計画する.
計画された経路に基づき,ローカルプランニングがロボットの動作をリアルタイムで制御し,動的な障害物や環境の変化に対応する.
センサデータは,動的なコストマップを更新し,環境の変化を即座に反映する.この仕組みにより,静的なマップに基づいたプランニングと動的な環境対応を統合し,安全かつ効率的な自律移動を実現する.
メトリックマップに基づくナビゲーションは,精密な環境情報を活用できる点が利点である.
しかし,高精度な地図やセンサ,計算資源を必要とするため,これらの要件に応じたシステム設計が求められる.