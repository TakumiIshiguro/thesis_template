\chapter{要素技術}
\section{メトリックマップに基づくナビゲーション}
メトリックマップに基づくナビゲーションについて説明する.
ナビゲーションを実現するためには,LiDARやオドメトリなどのセンサとメトリックマップを活用し,自己位置推定や経路計画を行うことで,ロボットが目的地まで自律的に移動する仕組みが必要となる.
まず,自己位置推定では,ロボットが地図上のどこに位置しているかを特定する.
これには,LiDARやオドメトリデータなどのセンサ情報を利用し,AMCL(Adaptive Monte Carlo Localization)などのアルゴリズムを活用する.
自己位置推定が成功することで,ロボットの現在位置が正確に把握される.
次に,自己位置から目的地までの最適な経路を計画する.計画された経路に基づき,ロボットの動作をリアルタイムで制御し,障害物や環境の変化にも対応する必要がある.
このようにして,メトリックマップに基づくプランニングと動的環境への対応を統合し,ロボットの自律移動を実現する.
メトリックマップに基づくナビゲーションの利点として,事前に取得した環境情報を有効活用できる点が挙げられる.
しかし,事前に取得した環境情報と現在の環境情報が大きく異なる場合,自律移動が失敗する可能性がある点が課題となる.
本論文ではメトリックマップに基づくナビゲーションを教師として模倣学習を行う.