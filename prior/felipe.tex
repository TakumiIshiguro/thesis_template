\section{条件付き模倣学習}
1 章でも述べたが,Felipe ら\cite{codevilla2018endtoend}は視覚を入力とした end-to-end 学習により自動運転を行う手法において,右折や左折といった行動をネットワークの入力に加えることで,性能が向上することを報告している.
Felipe らは 2 種類のネットワークを提案している.
\figref{fig:felipe_branched}に示す (a) のネットワークは画像を処理する CNN ,そして CNN の出力と観測,目標方向などのコマンドを入力とする全結合層で構成されている.
\figref{fig:felipe_branched}に示す (b) のネットワークは画像を処理する CNN ,そして CNN の出力と観測を目標方向などのコマンドによって全結合層を切り替える構造となっている.
シミュレータと実環境の両方で実験が行われており,どちらも (b) のネットワークがより優れた結果となった.

\begin{figure}[htbp]
  \centering
  \includegraphics[width=140mm]{images/pdf/other/branched.pdf}
  \caption[Two network architectures for command-conditional imitation learning]{Two network architectures for command-conditional imitation learning 
  \protect\linebreak (Quoted from\cite{codevilla2018endtoend})}
  \label{fig:felipe_branched}
\end{figure}

\clearpage