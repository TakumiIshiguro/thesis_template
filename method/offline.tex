\newpage
\section{オフライン学習}
春山らの先行研究では学習機の訓練に取得したデータを逐次学習するオンライン学習を用いていた.
オンライン学習の欠点として,学習するデータに偏りが発生してしまう点が挙げられる.
具体的には,学習の初期に取得したデータは学習される回数が多くなり,学習の後半に取得したデータは学習される回数が少ない.
これにより,学習が不十分な場合は経路追従できない箇所が生じることがある.
この欠点を補うために,オフライン学習を併用して行う.
% オンライン学習では学習データを収集しつつ,そのデータを学習に利用する一方で,オフライン学習は予め収集したデータを学習することですべてのデータを均等に学習できる.
オフライン学習とは一般的に用いられる学習方法で,予め収集したデータを使用して学習する手法を指す.
データを予め収集することにより,すべてのデータを同じ回数学習することが可能である.
本論文では,オンライン学習を最初に行い,そこで作成したデータセットを使用して学習することをオフライン学習と呼ぶ.
