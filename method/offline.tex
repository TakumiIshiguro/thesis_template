\newpage
\section{オフライン学習}
春山らの先行研究では学習機の訓練の手法にオンライン学習を用いていた.
オンライン学習の欠点として,学習するデータに偏りが発生してしまう点が挙げられる.
具体的には,学習の初期に取得したデータは学習される回数が多くなり,学習の後半に取得したデータは学習される回数が減少する.
これにより,学習が不十分な場合,経路追従できない箇所が生じることがある.
この欠点を補うために,オフライン学習を併用して行う.
% オンライン学習では学習データを収集しつつ,そのデータを学習に利用する一方で,オフライン学習は予め収集したデータを学習することですべてのデータを均等に学習できる.
オフライン学習とは一般的に用いられる学習方法で,予め収集したデータを使用して学習する手法を指す.
データを予め収集することにより,すべてのデータを同じ回数学習することが可能である.
今回のオフライン学習とは,オンライン学習を最初に行い,そこで作成したデータセットを使用して学習することを意味する.
