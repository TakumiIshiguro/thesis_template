\newpage
\section{オフライン学習}
春山らの先行研究では学習機の訓練の手法にオンライン学習を用いていた.
オンライン学習の欠点として,学習するデータに偏りが発生してしまう点が挙げられる.
具体的には,学習の初期に取得したデータは学習される回数が多くなり,学習の後半に取得したデータは学習のされる回数は減少する.
これにより,学習が不十分な場合,経路追従できない箇所が存在する可能性がある.
この欠点を補うために,オフライン学習を併用して行うことを提案する.
% オンライン学習では学習データを収集しつつ,そのデータを学習に利用する一方で,オフライン学習は予め収集したデータを学習することですべてのデータを均等に学習できる.
オフライン学習とは一般的に用いられる学習方法で,予め収集したデータを使用して学習する手法を指す.
データを予め収集することにより,すべてのデータを同じ回数学習することが可能である.
また,オンライン学習では学習するために走行する必要があるが,オフライン学習を併用することで追加の学習を走行せずに行うことができる.