\chapter{機能の改善}
\label{chap:method}
% \section{新たななルート}
春山らは以下図左の経路を対象として実験を行っている.
しかし,今回は春山らの実験では対象としていないシナリオにも対応するため以下の図右の領域を実験対象に追加する.

図を追加する



新たな領域のデータを収集するために以下のようなルートを走行した

ルートの図



\subsection{ネットワークの変更}
春山らの先行研究では,~~に示すネットワークを使用していた.
一方でfelipeらの先行研究によると,~~に示す,出力層がコマンドによって分岐する形式のネットワークがより経路追従の成功率が向上すると報告している.
そのため,今回の研究ではfelipeらによって提案されたネットワークを採用した.
\subsection{オフライン学習}
春山らの先行研究では学習機の訓練の手法はオンライン学習を用いていた.
オンライン学習の欠点として,学習するデータに偏りが発生してしまう.
一般的に学習の初期に取得したデータは学習される回数が多くなり,学習の後半に取得したデータは学習のされる回数は減少する.
このため,経路追従できない箇所が発生してしまう可能性がある.
今回の実験では,この欠点を補うために,オフライン学習を併用して行う.
% オンライン学習では学習データを収集しつつ,そのデータを学習に利用する一方で,オフライン学習は予め収集したデータを学習することですべてのデータを均等に学習できる.
オフライン学習とは一般的に用いられる学習方法で,予め収集したデータを使用して学習する手法を指す.
データを予め収集することにより,すべてのデータを均等に学習することができる.
\section{経路追従モジュール}

\section{通路分類モジュール}

