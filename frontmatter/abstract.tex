%!TEX root = ../thesis.tex
\chapter*{概要}
\thispagestyle{empty}
%
\begin{center}
  \scalebox{1.0}{視覚と行動のend-to-end学習により}\\
  \vspace{-1.0zh}
  \scalebox{1.0}{経路追従行動を模倣する手法の提案}\\
  \scalebox{1.0}{-経路選択の成功率向上を意図したネットワークの変更と実験的評価-}\\
\end{center}
\vspace{1.0zh}

本研究室では,移動ロボットのナビゲーション手段を冗長化するために,いくつかの手法を提案している.
岡田らはメトリックマップに基づくナビゲーションから生成される経路追従行動を end-to-end 学習を用いて,視覚を入力とする行動にオンラインで模倣する手法を提案してきた.
春山らは,岡田らの手法に加えて,カメラ画像から通路の種類を分類,シナリオによって目標方向を決定し,経路を選択する機能を追加している.
ここでのシナリオとは,島田らが提案した,「条件」と「行動」に関する単語を組みわせて構成された 50 例の文章を指す.
% 実ロボットを用いた実験から,構築したシステムにより視覚に基づいて経路を追従して目的地へ到達可能であることを確認した.
春山らは,島田ら\cite{Shimada2020}が作成したシナリオから 7 例を選定し,そのすべてで視覚に基づいて経路を追従して目的地まで到達できることを確認している.
選定外のシナリオでは地面の色が変化したり広場などを通過したりするため,視覚に基づいて経路追従するのがより困難な環境の可能性がある.

そのため本論文では,春山らが対象としていないシナリオでも,目的地までカメラ画像のみを入力として経路追従できるか調査する.
% 失敗する場合は要因を調査することで,システムの改良点について考察できるようになる.
はじめに,経路追従の成功率を高めるために,ネットワークの変更や新たな学習方法を導入した.
シミュレータを用いた実験によって,これらの手法が経路追従の成功率を上昇させるか検証した.
さらに,実ロボットを用いた実験によって,春山らが対象としていないシナリオでも目的地まで移動できることを確認した.

キーワード: 自律移動ロボット end-to-end 学習 ナビゲーション
%
\newpage
%%
\chapter*{abstract}
\thispagestyle{empty}
%
\begin{center}
  \scalebox{1.0}{A proposal for an imitation method of path-tracking behavior}\\
  \vspace{-1zh}
  \scalebox{1.0}{by end-to-end learning of vision and action}
  \scalebox{0.9}{-Modification of network and experimental evaluation aimed at improving}\\
  \vspace{-1zh}
  \scalebox{0.9}{route selection success rate}
\end{center}
\vspace{1.0zh}
いったん後回し
% In our laboratory, several methods have been proposed to enhance the redundancy of navigation approaches for mobile robots.
% Okada et al. proposed a method that uses end-to-end learning to imitate path-following behavior generated from metric map-based navigation and applies it to vision-based behavior.
% Haruyama et al. extended Okada et al.'s approach by adding functionality to recognize intersection from camera images, determine target directions based on scenarios, and select routes. 
% Here, "scenarios" refer to 50 sentences proposed by Shimada et al., consisting of combinations of words related to "conditions" and "actions."
% Experiments with real robots demonstrated that the constructed system enables destination arrival through vision-based path-following. 
% However, Haruyama et al. showed that only 7 out of the 50 scenarios created by Shimada et al. were successfully autonomously navigated, as the experimental area was partially limited.

% Therefore, this paper investigates whether autonomous navigation to the destination can be achieved using only camera images, even in areas not targeted.
% First, network modifications and new training methods were implemented to improve the success rate of path-following. 
% The effectiveness of these approaches was evaluated through experiments in a simulator. 
% Furthermore, experiments with real robots confirmed that the system could navigate to destinations in scenarios that include unverified areas.

keywords: autonomouse moblie robot, end-to-end learning, navigation
