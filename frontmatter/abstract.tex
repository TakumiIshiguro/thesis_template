%!TEX root = ../thesis.tex
\chapter*{概要}
\thispagestyle{empty}
%
\begin{center}
  \scalebox{1.5}{タイトル}\\
\end{center}
\vspace{1.0zh}

本研究室では,移動ロボットのナビゲーション手段を冗長化するために,いくつかの手法を提案している.
岡田らはメトリックマップに基づくナビゲーションから生成される経路追従行動を end-to-end 学習を用いて,視覚を入力とする行動にオンラインで模倣する手法を提案してきた.
春山らは,岡田らの手法に加えて,カメラ画像から分岐路を認識,シナリオによって目標方向を決定し,経路を選択する機能を追加している.
ここでのシナリオとは,島田らが提案した,「条件」と「行動」に関する単語を組みわせて構成された50例の文章を指す.
実ロボットを用いた実験から,構築したシステムにより視覚に基づいて経路を追従して目的地へ到達可能であることを確認した.
一方で,春山らは実験の対象としているエリアが部分的であるため,島田らが作成したシナリオ50例中の7例のみ自律移動が確認されている.

そのため本論文では,これまで対象としていないエリアを含む場合でも,目的地までカメラ画像のみを入力として自律移動できるか調査する.
% 失敗する場合は要因を調査することで,システムの改良点について考察できるようになる.
また,経路追従の成功率を高めるために,ネットワークの変更や新たな学習方法を行った.
実ロボットを用いた実験によって,これまで自律移動移動が確認されていないエリアを含むシナリオでも目的地まで移動できることを確認した.

キーワード: 
%
\newpage
%%
\chapter*{abstract}
\thispagestyle{empty}
%
\begin{center}
  \scalebox{1.3}{title}
\end{center}
\vspace{1.0zh}
%


keywords: 
