\chapter{おわりに}
\label{chap:end}
\section{結論}
本論文では,春山らが提案したシステムを改良し,先行研究では走行が未確認であったシナリオに対しても目的地までカメラ画像のみを入力として自律移動可能か調査した.
経路追従の成功率を向上させることを目的として,先行研究からは,行動ごとにモデルを分けるネットワークに変更,オフライン学習による追学習する仕組みを追加した.
また,これらの手法が有効であることをシミュレータでの実験で検証した.
実ロボットを用いた実験では,先行研究では走行が確認されていないエリアをシナリオを含む場合でも 28 例中,24例は自律移動が可能であること確認した.
失敗した4例に関して,通路の特徴の分類が遅れることによって,経路追従に失敗すると考えられる.